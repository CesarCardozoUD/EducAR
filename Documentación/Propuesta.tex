\documentclass[12pt,letterpaper]{article}
\usepackage[utf8]{inputenc}
\usepackage{amsmath}
\usepackage{amsfonts}
\usepackage{amssymb}
\author{Cesar Cardozo}
\title{EducAR: Ambiente Virtual de Aprendizaje para Niños de Preescolar Usando Realidad Aumentada}
\begin{document}
\maketitle
\section{Panorama General: Derechos Básicos de Aprendizaje (DBA)}
“Los DBA son el conjunto de aprendizajes estructurantes que construyen las niñas y los niños a través de las interacciones que establecen con el mundo, con los otros y consigo mismos, por medio de experiencias y ambientes pedagógicos en los que está presente el juego, las expresiones artísticas, la exploración del medio y la literatura.”\footnote{Definición extraida de \textit{“DBA Transición”, Gobierno de Colombia. Extraido de: [ https://aprende.colombiaaprende.edu.co/ckfinder/userfiles/files/DBA0\%20Transici\%C3\%B3n.pdf]}}\\

Los DBA proporcionan una pauta brindada por el gobierno de Colombia, a traves del ministerio de educación, sobre la cual los docentes deben trabajar para garantizar los derechos minimos de aprendizaje que se le debe proporcionar a cada estudiante.\\

"... son una herramienta para construir estrategias que permitan la continuidad y articulación de los procesos que viven las niñas y los niños en su paso grado a grado en el entorno educativo; aportan en la construcción de acuerdos sobre aquello que deben aprender y a la complejización de los aprendizajes que desarrollarán en su vida escolar."\footnote{ibid}\\

Los DBA correspondientes al nivel de preescolar tienen como fin, guiar al maestro para que genere un ambiente afable en el entorno escolar con el fin de que los niños se sientan comodos y gustosos en su aprendizaje, para este objetivo los DBA se dividen en 3 lineamientos:

\begin{itemize}
\item Las niñas y los niños construyen su identidad en relación con los otros; se sienten queridos, y valoran positivamente pertenecer a una familia, cultura y mundo.
\item Las niñas y los niños son comunicadores activos de sus ideas, sentimientos y emociones; expresan, imaginan y representan su realidad.
\item Las niñas y los niños disfrutan aprender; exploran y se relacionan con el mundo para comprenderlo y construirlo.
\end{itemize}

\section{Resumen Ejecutivo}
\subsection{¿Que Problema Hemos Encontrado?}
\rule{150mm}{0.1mm} 
En Colombia no se poseen cifras exactas de las personas que padecen este trastorno especifico de lenguaje (\textit{TEL}), sin embargo en el panorama mundial se estima que entre el 10 y el 17.5\% de las personas deben vivir con este trastorno; esto se agrava por la mala comprensión y diagnosis de este \textit{TEL} que resulta en una grave afectación en el autoestima de los niños que lo padecen debido a la erronea costumbre de ligarla a las capacidades cognitivas de los niños. \\

\subsection{¿En que Consiste el Proyecto?}
\rule{150mm}{0.1mm} 
Nuestro proyecto consiste en hacer uso de las tecnologías emergentes como lo son la realidad virtual (\textit{VR}) y la realidad aumentada (\textit{AR}) con el fin de ejercitar la lateralidad y aspectos comunes en la sintomatología de la dislexía mediante minijuegos que resulten llamativos y sobre todo divertidos para los niños.\\

\subsection{¿Cuales son las Fuentes de Ingreso?*}
\rule{150mm}{0.1mm} 
La fuente de ingresos para la empresa será la venta de licencias de uso para entidades que esten interesadas en la adquisición del producto para el tratamiento de los  niños con este \textit{TEL}.\\

\subsection{¿Quienes son los Emprendedores?***}
\rule{150mm}{0.1mm} 
Los impulsores de este proyecto somos:
Rodrigo Alvarez y Cesar Cardozo, ambos estudiantes activos de la Universidad Distrital Francisco José de Caldas en el proyecto curricular de Ingeniería de Sistemas.\\

\subsection{¿Que experiencia tenemos?****}
\rule{150mm}{0.1mm} 
Con materías cursadas a lo largo de nuestra carrera universitaria como \textbf{Aprendizaje y problemas del aprendizaje} y con experiencia adquirida durante nuestro paso por los grupos de trabajo/investigación como \textbf{Virtus} y \textbf{Dev Games}, sumado a la experiencia del equipo en desarrollo de videojuegos llegando incluso a participar de eventos como el SOFA en varias ocaciones, estamos suficiente capacitados como para formular, diseñar, desarrollar e implementar dicho proyecto con base a investigaciones propias y externas.\\

\subsection{¿Cual es la inversión a realizar?}
\rule{150mm}{0.1mm} 
Una de las mayores ventajas de este proyecto en particular es que solo requiere de equipos y talento humano; ambas suplidas por la empresa.
La unica Inversión que se debe realizar es de tiempo dados los matices de dificultad de tratar con un Trastorno Especifico de Lenguaje.


\subsection{¿Por que Creemos que el Proyecto va a Funcionar?}
\rule{150mm}{0.1mm} 
En diversas ocaciones se han generado acercamientos para tratar no solo este sino una amplia gama de \textit{TEL} alrededor del mundo, siendo uno de los más exitosos el desarrollado por la Phd. Luz Rello que implementa mecanicas de gamificación e inteligencia computacional para la diagnosis de la Dislexía; este proyecto a tenido una gran acogida en España y el mundo entero dada la presición de la diagnosis de la Dislexía y la interactividad, que si bien se puede volver aún más inmersiva (Como lo pretende el presente proyecto), los niños adoran.

Como es de costumbre en los nuevos acercamientos al tratamiento de enfermedades, se deberá tener precaución y prudencia antes de decir que estos minijuegos en verdad funcionan como "tratamiento" de este \textit{TEL} por lo que se deberán llevar estudios adicionales posteriores a la implementación del mismo, sin embargo todas las investigaciones \textit{a priori} que se han realizado parecen indicar que este tipo de ejercicios (refuerzos de lateralidad y procesamiento) suelen ser una terapia efectiva.

\section{Objetivos}
\subsection{Objetivo General}
\rule{150mm}{0.1mm} 
Desarrollar un aplicativo para ejercitar y reforzar la lateralidad de los niños que padezcan de Dislexía haciendo uso de herramientas de inmersión tecnologica como el \textit{AR} y \textit{VR} que pueda ser utilizado como alternativa a las terapias convensionales y que resulte atractivo y divertido a los pacientes.\\

\subsection{Objetivos Especificos}
\rule{150mm}{0.1mm} 
\begin{itemize}
\item Formular un abordaje viable para el tratamiento de la Dislexía a partir de mecanicas entretenida, interactivas e inmersivas.
\item Diseñar "Minijuegos" que se acoplen a los tratamientos formulados  y que se adecuen al cuadro clinico de los pacientes.
\item Implementar las tecnologías de \textit{AR} y  \textit{VR} de manera armonica con el contenido desarrollado.
\item Modelar y desarrollar un demo del aplicativo que permita validar y evaluar la idea propuesta.
\end{itemize}


\end{document}